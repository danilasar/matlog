\chapter{9 февраля. Введение в математическую логику}
\section{Предмет математической логики}
\dftion Логика --- анализ принципов правильных суждений. Она возникла в VI--IV вв. до н. э. Основоположник логики --- Аристотель, изложивший идею дедуктивного подхода в <<Аналитике>>.

\dftion Формальная логика изучает формы, в которых проявляются законы причинно-следственных связей вне зависимости от содержания (смысла) тех явлений (предметов), к которым эти законы относятся.

\dftion Математическая логика занимается обоснованием правильных способов рассуждений математического аппарата.

Главная цель математической логики --- формализация и обоснование правильных способов \textbf{математических рассуждений} с целью точного определения понятия <<математическое доказательство>>.

\section{Этапы развития математической логики}
Дж. Буль создал алгебру логики.

Г. Фреге разработал логико-математические языки и теорию их осмысления (семантику).

Д. Гильберт разработал программу обоснования математики на основе \textbf{аксиоматического подхода}.

\section{Задача математической логики}
Изучение фундаментальных теорий, представляющих собой множества теорем, получающихся из исходных аксиом с помощью дедуктивных умозаключений.

Проблемы: непротиворечивость, полнота и разрешимость теорий.

Проблема разрешимости теорий --- первоисточник теории алгоритмов!

\section{Логика высказываний}
\textit{Высказывание} --- повествовательное предложение, о котором можно судить, истинное оно или ложное. Обозначают высказывания: $A, B, C$...

\textit{Истинностное значение} высказывания $A$ обозначется символом $\lambda(A)$ и определяется по формуле:
$\lambda(A) = 1$, если $A$ истинно, и $\lambda(A) = 0$ в обратном случае.

\section{Алгебра высказываний. Формулы алгебры высказываний}
Алгебра выражений задаётся операциями $\lnot$ (<<не>>), $\land$ (<<и>>), $\lor$ (<<или>>), $\then$ (<<следует>>), $\eq$ (<<равносильно>>) над множеством $P$:

$\mathscr{P} = (\mathscr{P}, \lnot, \land, \lor, \then, \eq)$

В программировании также распространены <<исключающее и>> и <<исключающее или>>, но эти операции не входят в базовый набор.

Отрицание $\lnot A$ истинно тогда и только тогда, когда $A$ ложно. Таблица истинности отрицания:

\begin{tabular}{|c|c|}
    \hline
    $A$ & $\lnot A$ \\
    \hline
    0   &  1        \\
    \hline
    1   &  0        \\
    \hline
\end{tabular}

Конъюнкция $a \land B$ (<<A и B>>) истинна тогда и только тогда, когда оба высказывания истинны.

Дизъюнкция $A \lor B$ (<<A или B>>) ложна тогда и только тогда, когда оба высказывания ложны.

Импликация $A \then B$ (<<если A, то B>>, <<из A следует B>>, <<A необходимо для B>>) ложна тогда и только тогда, когда $A$ истинно, а $B$ ложно. $A$ называется посылкой, а $B$ --- следствием.

\dftion Свойства алгебры высказываний $\mathscr{P}$ описываются с помощью формул, которые строятся из переменных символов с помощью знаков логических операций. Такие формулы принято называть также \textbf{пропозициональными формулами}.

\dftion Символы логических операций называются \textbf{пропозициональными связками}.

\dftion Переменные символы $X, Y, Z, \dots$, которые используются для обозначения высказываний, называются \textbf{пропозициональными переменными}.

\dftion Формулы алгебры высказываний индуктивно определяются по правилам:
\begin{enumerate}
    \item Каждая пропозициональная переменная является формулой
    \item Если $\Phi, \Psi$ --- формулы, то формулами также являются выражения $(\lnot \Phi)$, $(\Phi \land \Psi)$, $(\Phi \lor \Psi)$, $(\Phi \then \Psi)$, $(\Phi \eq \Psi)$.
\end{enumerate}

Множество всех формул алгебры высказываний обозначим $\mathfrak I$.

Приоритет логических операций:
\begin{enumerate}
    \item Отрицание
    \item Конъюнкция
    \item Дизъюнкция
    \item Импликация и эквивалентность (имеют равный приоритет)
\end{enumerate}

Если в формулу $\Phi$ входят переменные $X_1, \dots, X_n$, то записывают $\Phi = \Phi(X_1, \dots, X_n)$.

Формула $\Phi$ определяет функцию $n$ переменных $F_\Phi$, которая каждому упорядоченному набору $(\lambda(X_1), \dots, \lambda(X_n))$  n элементов множества $\{0, 1\}$ ставит в соответствие элемент $\lambda(\Phi(X_1, \dots, X_n))$ этого же множества.

\dftion $F_\Phi$ --- \textit{истинностная} функция формулы $\Phi$. Графически представляется истинностной таблицей. Такая таблица содержит $2^n$ строк и имеет одно из $2^{2^n}$ возможных распределений значений $0$ и $1$ в последнем столбце.

$F_\Phi: \{0, 1\} \then \{0, 1\}$ --- истинностная функция формулы $\Phi$.

$F_\Phi(\lambda(A_1), \dots, \lambda(A_n)) = \lambda(\Phi(A_1, \dots, A_n)) \in \{0, 1\}$

Истинностная таблица формулы $\Phi$:

\begin{tabular}{|ccccc|}
    \hline
    ~         & $X_1$    & $\cdots$ & $X_n$    & $\Phi(X_1, \cdots, X_n)$ \\
    \hline
    0         & 0        & $\cdots$ & 0        & $k_0$ \\
    1         & 0        & $\cdots$ & 1        & $k_1$ \\
    $\vdots$  & $\vdots$ & $\ddots$ & $\vdots$ & $\vdots$ \\
    $2^n - 1$ & 1        & $\cdots$ & 1        & $k_{2^n - 1}$ \\
    \hline
\end{tabular}

Всего в такой тоблице $2^n$ строк и $2^{2^n}$ истинностных функций (распределений значений 0 и 1 в последнем столбце) от $n$ переменных.

\ex Формула $\Phi=(\lnot X \land \lnot Y \Leftrightarrow X \lor \lnot Y)$ имеет следующую истинностную таблицу:

\begin{tabular}{|ccccccc|}
    \hline
    $X$ & $Y$ & $\lnot X$ & $\lnot Y$ & $\lnot X \land \lnot Y$ & $X \lor \lnot Y$ & $\lnot X \land \lnot Y \eq X \lor \lnot Y$ \\
    \hline
    0 & 0 & 1 & 1 & 1 & 1 & 1 \\
    0 & 1 & 1 & 0 & 0 & 0 & 1 \\
    1 & 0 & 0 & 1 & 0 & 1 & 0 \\
    1 & 1 & 0 & 0 & 0 & 1 & 0 \\
    \hline
\end{tabular}

\ex Составим таблицу истинности для формулы $(P \overset 1 \Rightarrow Q) \overset 5 \Leftrightarrow (\overset 2 \lnot Q \overset 4 \Rightarrow \overset 3 \lnot P)$:

\begin{tabular}{|ccccccc|}
    \hline
    $p$ & $Q$ & 1 & 2 & 3 & 4 & 5 \\
    \hline
    0 & 0 & 1 & 1 & 1 & 1 & 1 \\
    0 & 1 & 0 & 0 & 1 & 1 & 0 \\
    1 & 0 & 1 & 1 & 0 & 0 & 0 \\
    1 & 1 & 1 & 0 & 0 & 1 & 1 \\
    \hline
\end{tabular}

\dftion Формула $\Phi$ называется:
\begin{enumerate}
    \item \textit{тавтологией} (или тождественно истинной формулой) и обозначется $|= \Phi$, если её истинностная функция равна 1;
    \item \textit{противоречием} (или тождественно ложной формулой), если её истинностная функция тождественно равна 0;
    \item \textit{выполнимой}, если её истинностная функция не равна тождественно 0;
    \item \textit{опровержимой}, если её истинностная функция не равна тождественно 1.
\end{enumerate}

\dftion Тавтологии являются общими схемами построения истинных высказываний и в этом смысле выражают определённые \textit{логические законы}.

Новые тавтологии можно получить с помощью правила подстановки:

\textbf{Правило подстановки}. Если $\vDash \Phi(X_1, \dots, X_n)$, то для любых формул $\Phi_1, \dots, \Phi_n$ выполняется $\vDash \Phi(\Phi_1, \dots, \Phi_n)$

Формулы $\Phi, \Psi$ логически равносильны, если $F_\Phi \equiv F_\Psi$, т. к. $|= \Phi \eq \Psi$.

Обозначим $\Phi = \Psi$, $\Phi \hat{=} \Psi$. На $\mathfrak{I}_{AB}$ определно отношение эквивалентности $\cong$.

\section{Лемма об основных равенствах формул}

\begin{enumerate}
    \item $X \lor (Y \lor Z) = (X \lor Y) \lor Z$ --- сочетательный закон;
    \item $X \lor Y = Y \lor X$ ($X \land Y = Y \land X$) --- перестановочный закон;
    \item $X \lor X = X$ ($X \land X = X$);
    \item $X \land (Y \lor Z) = (X \land Y) \lor (X \land Z)$ ($X \lor (Y \land Z) = (X \lor Y) \land (X \lor Z)$);
    \item $\lnot(X \land Y) = \lnot X \lor \lnot Y$ ($\lnot(X \lor Y) = \lnot X \land \lnot Y$) --- закон де Моргана;
    \item $(X \land Y) \lor X = X$ ($(X \lor Y) \land X = X$) --- закон поглощения;
    \item $\lnot \lnot X$ --- закон двойного отрицания;
    \item $X \Rightarrow Y = \lnot X \lor Y = \lnot(X \land \lnot Y)$;
    \item $X \Leftrightarrow Y = (X \Rightarrow Y) \land (Y \Rightarrow X)$
    
    $X \Leftrightarrow Y = (\lnot X \lor Y) \land (\lnot Y \lor X)$

    $X \Leftrightarrow Y = (X \land Y) \lor (\lnot X \land \lnot Y)$
\end{enumerate}

\dftion \textit{Литера} --- это переменная или её отрицание:

\begin{equation*}
    X^\alpha = 
        \left\{ \begin{aligned}
        x, \, \text{если} \, \alpha = 1 \\
        \lnot x, \, \text{если} \, \alpha = 0
        \end{aligned} \right.
\end{equation*}

\dftion \textit{Дизъюнктивная нормальная форма (ДНФ)} --- это дизъюнкция конъюнкций $(X \land Y) \lor (X \land Z)$

\ex $(X \Leftrightarrow \lnot Y) \lor \lnot (X \Rightarrow Y) \overset {\text{ДНФ}}{=}$ $(X \land \lnot Y) \lor$ $(\lnot X \land Y) \lor$ $(X \land \lnot Y)) =$ $(X \land \lnot Y) \lor$ $(\lnot X \land Y)$

\dftion \textit{Конъюнктивная нормальная форма (КНФ)} --- это конъюнкция дизъюнктов или один дизъюнкт.

\dftion \textit{Совершенная дизъюнктивная нормальная форма (СДНФ)} --- это ДНФ, в которой все конъюнкты соверщенны, то есть содержат все переменные этой формулы.

\dftion \textit{Совершенная конъюнктивная нормальная форма (СКНФ)} --- это конъюнктивная нормальная форма (КНФ), в которой все дизъюнкты соверщенны, то есть содержат все переменные этой формулы.

СДНФ и СКНФ вычисляются очень просто на основании истинностных таблиц.

\textbf{Теорема СДНФ}. Любая выполнимая формула $\Phi$, у которой $n$ переменных, логически равносильная формуле вида $\bigvee(x_1^{\alpha_1} \land \dots \land x_n^{a_n})$.

\textbf{Теорема СКНФ}. Любая выполнимая формула $\Phi$, у которой $n$ переменных, логически равносильная формуле вида $\bigwedge(x_1^{\alpha_1} \land \dots \land x_n^{a_n})$.