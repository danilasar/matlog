\chapter{16 февраля. Логическая равносильность формул}
\section{Логическая равносильность формул}
\dftion Формулы $\Phi, \Psi$ называются \textbf{логически равносильными} (или просто равносильными), если они принимают одинаковые логические значения при любых истинностных значениях их переменных. Это равносильно условию $\vDash \Phi \eq \Psi$.

Для обозначения логически эквивалентных формул используется символическая запись $\Phi = \Psi$ или $\Phi \cong \Psi$.

$\Phi = \Psi$ однознач. $\vDash \Phi \eq \Psi$

Такие выражения называются \textbf{логическими равенствами} или просто \textit{равенствами формул}.

\textbf{Лемма 1}. Справедливы следующие равенства формул:
\begin{enumerate}
    \item $X \lor (Y \lor Z) = (X \lor Y) \lor Z$ --- свойство ассоциативности дизъюнкции и конъюнкции;
    \item $X \lor Y = Y \lor X$ ($X \land Y = Y \land X$) --- свойство коммутативности дизъюнкции и конъюнкции;
    \item $X \lor X = X$ ($X \land X = X$) --- свойство идемпотентности;
    \item $X \land (Y \lor Z) = (X \land Y) \lor (X \land Z)$ ($X \lor (Y \land Z) = (X \lor Y) \land (X \lor Z)$) --- законы дистрибутивности конъюнкции относительно дизъюнкции и дизъюнкции относительно конъюнкции;
    \item $\lnot(X \land Y) = \lnot X \lor \lnot Y$ ($\lnot(X \lor Y) = \lnot X \land \lnot Y$) --- законы де Моргана;
    \item $(X \land Y) \lor X = X$ ($(X \lor Y) \land X = X$) --- законы поглощения;
    \item $\lnot \lnot X$ --- закон двойного отрицания;
    \item $X \Rightarrow Y = \lnot X \lor Y = \lnot(X \land \lnot Y)$ --- взаимосвязь импликации с дизъюнкцией и конъюнкцией;
    \item $X \Leftrightarrow Y = (X \Rightarrow Y) \land (Y \Rightarrow X) = (\lnot X \lor Y) \land (\lnot Y \lor X) = (X \land Y) \lor (\lnot X \land \lnot Y)$ --- взаимосвязь равносильности с дизъюнкцией и конъюнкцией
\end{enumerate}

\textbf{Лемма (правило замены)}. Если формулы $\Phi, \Phi'$ равносильны, то для любой формулы $\Psi(X)$, содержащей переменную $X$ выполняется равенство $\Psi(\Phi)=\Psi(\Phi')$.

Это правило означает, что при замене в любой формуле $\Psi = \Psi(\Phi)$ некоторой её подформулы $\Phi$ на равносильную ей формулу $\Phi$ получается формула $\Psi = \Psi(\Phi)$, равносильная исходной формуле $\Psi$.

Такие переходы называются \textit{равносильными преобразованиями формул}.

\section{Нормальные формулы}
Из основных равенств следует, что для каждой формулы $\Phi \in F_{AB}$ можно указать равносильные ей формулы специального вида, содержащие только символы логических операций $\lnot, \land, \lor$.

\dftion \textbf{Литерой} называется пропозициональная переменная $X$ или её отрицание $\lnot X$. Для обозначения литеры используется символ $X^\alpha$, где $\alpha \in \{0,1\}$ и по определению $X^1 = X$, $X^0 = \lnot X$.

\dftion \textbf{Конъюнктом} (\textbf{дизъюнктом}) называется конъюнкция (дизъюнкция) литер или одна литера.

\dftion \textbf{Конъюнктивной нормальной формой} (КНФ) называется конъюнкция дизъюнктов или один дизъюнкт. Пример ДНФ: $D_1 \land D_2 \land \dots \land D_m$, где $D_i$ --- дизъюнкты.

\textbf{Дизъюнктивной нормальной формой} (ДНФ) называется дизъюнкция конъюнктов или один конъюнкт.

При этом КНФ (ДНФ) называется \textit{совершенной}, если все её дизъюнкты (конъюнкты) содержат все пропозициональные переменные рассматриваемой формулы.

\textbf{Теорема 1}. Любая формула равносильна некоторой ДНФ и некоторой КНФ.

\underline{Алгоритм} приведения формулы $\Phi$ к ДНФ (КНФ):
\begin{enumerate}
    \item выражаем все входящие в формулу $\Phi$ импликации и эквивалентности через конъюнкцию, дизъюнкцию и отрицание;
    \item согласно законам де Моргана все отрицания, стоящие перед скобками, вносим в эти скобки и сокращаем все двойные отрицания;
    \item согласно законам дистрибутивности преобразуем формулу так, чтобы все конъюнкции выполнялись раньше дизъюнкций (или дизъюнкции выполнялись раньше конъюнкций)
\end{enumerate}

\textbf{Теорема 2}. Любая выполнимая формула $\Phi = \Phi(X_1, \dots, X_n)$ равносильна формуле вида $\bigvee\limits_{\alpha_1,\dots,\alpha_n}(X_1^{\alpha_1} \land \dots \land X_n^{\alpha_n})$, где дизъюнкция берётся по всем упорядоченным наборам $(\alpha_1, \dots, \alpha_n) \in \{0,1\}^n$, удовлетворяющим условию $F_\Phi(\alpha_1, \dots, \alpha_n) = 1$.

Такая формула определяется однозначно (с точностью до порядка членов дизъюнкций и конъюнкций) и называется совершенной дизъюнктивной нормальной формулой (СДНФ) формулы $\Phi$.

\textbf{Теорема 3}. Любая выполнимая формула $\Phi = \Phi(X_1, \dots, X_n)$ равносильна формуле вида $\bigwedge\limits_{\alpha_1,\dots,\alpha_n}(X_1^{\alpha_1} \lor \dots \lor X_n^{\alpha_n})$, где конъюнкция берётся по всем упорядоченным наборам $(\alpha_1, \dots, \alpha_n) \in \{0,1\}^n$, удовлетворяющим условию $F_\Phi(\alpha_1, \dots, \alpha_n) = 1$.

Такая формула определяется однозначно (с точностью до порядка членов дизъюнкций и конъюнкций) и называется совершенной конъюнктивной нормальной формулой (СКНФ) формулы $\Phi$.

\underline{Алгоритм нахождения СДНФ и СКНФ} формулы $\Phi=\Phi(X_1,\dots,X_n)$:
\begin{enumerate}
    \item Составить истинностную таблицу формулы $\Phi$ и добавить два столбца <<Совершенные конъюнкты>> и <<Совершенные дизъюнкты>>
    \item Если при значениях $\lambda(X_1) = k_1, \dots, \lambda(X_n) = k_n$ значение $\lambda(\Phi(X_1,\dots,X_n))$ формулы $\Phi$ равно 1, то в соответствующей строке таблицы в столбце <<Совершенные конъюнкты>> записываем конъюнкт $X_1^{k_1} \land \dots \land X_n^{k_n}$ и в столбце <<Совершенные дизъюнкты>> делаем прочерк. При этом $X_i^1 = X_i$ и $X_i^0 = \lnot X_i$.
    \item Если при значениях $\lambda(X_1) = k_1, \dots, \lambda(X_n) = k_n$ значение $\lambda(\Phi(X_1,\dots,X_n))$ формулы $\Phi$ равно 0, то в соответствующей строке таблицы в столбце <<Совершенные дизъюнкты>> записываем дизъюнкт $X_1^{1-k_1} \lor \dots \lor X_n^{1-k_n}$ и в столбце <<Совершенные конъюнкты>> делаем прочерк. При этом $X_i^1 = X_i$ и $X_i^0 = \lnot X_i$.
    \item СДНФ равна дизъюнкции совершенных конъюнктов
    \item СКНФ равна конъюнкции совершенных дизъюнктов
\end{enumerate}

\section{Логическое следование формул}
\dftion Формула $\Phi$ называется \textbf{логическим следствием} формул $\Phi_1, \dots, \Phi_n$, если при любой подстановке в эти формулы вместо их переменных $X_1, \dots, X_n$ конкретных высказываний $A_1, \dots, A_n$ из истинности высказываний $\Phi_1(A_1, \dots, A_n), \dots, \Phi_m(A_1, \dots, A_n)$ следует истинность высказывания $\Phi(A_1, \dots, A_n)$.

Символическое обозначение $\Phi_1, \dots, \Phi_n \vDash \Phi$ называется \textbf{логическим следованием}. Формулы $\Phi_1, \dots, \Phi_n$ называются \textit{посылками} и формула $\Phi$ --- следствием логического следования $\Phi_1, \dots, \Phi_n \vDash \Phi$.

$1 \vDash \Phi$ или $\vdash \Phi$ (\textit{логическая тавтология}) --- частный случай логического следствия.

$\Phi \vDash$ или $\Phi \vDash 0$ --- формула $\Phi$ является противоречием.

\subsection{Основные правила логического следования}
\begin{enumerate}
    \item Правило отделения (\textit{modus ponens}): $$\Phi, \Phi \then \Psi \vDash \Psi$$
    \item Правило контрапозиции: $$\Phi \then \Psi \vDash \lnot \Psi \then \lnot \Phi$$
    \item Правило цепного заключения: $$\Phi_1 \then \Phi_2, \Phi_2 \then \Phi_3 \vDash \Phi_1 \then \Phi_3$$
    \item Правило перестановки посылок: $$\Phi_1 \then (\Phi_2 \then \Phi_3) \vDash \Phi_2 \then (\Phi_1 \then \Phi_3)$$
\end{enumerate}

\dftion Множество формул $\Phi_1, \dots, \Phi_m$ называется \textit{противоречивым}, если из него логически следует любая (в том числе ложная) формула $\Phi$. Символически это записывается $\Phi_1, \dots, \Phi_m \vDash$. В противном случае множество формул $\Phi_1, \dots, \Phi_m$ называется \textit{выполнимым}.

\textbf{Лемма (Критерии логического следования)}. Условие $\Phi_1, \dots, \Phi_m \vDash \Phi$ равносильно каждому из следующих условий:
\begin{itemize}
    \item $\Phi_1 \land \dots \land \Phi_n \vDash \Phi$
    \item $\vDash \Phi_1 \land \dots \Phi_n \then \Phi$
    \item $\Phi_1 \land \dots \Phi_n, \lnot \Phi \vDash$
\end{itemize}
В частности, $\Phi \vDash \Psi$ равносильно $\vDash \Phi \then \Psi$.

\textbf{Вывод}. Следующие задачи равносильны:
\begin{enumerate}
    \item Проверка тождественной истинности
    \item Проверка логического следования
    \item Проверка тождественной ложности
    \item Проверка противоречивость
\end{enumerate}

\section{Методы проверки тождественной истинности формул}

Основные методы проверки тождественной истинности формул:
\begin{enumerate}
    \item Прямой метод
    \item Алгебраический метод
    \item Алгоритм Квайна
    \item Алгоритм редукции
    \item Метод семантических таблиц
    \item Метод резолюций
\end{enumerate}

\subsection{Алгебраический метод}
Преобразование формулы $\Phi = \Phi(X_1, \dots, X_n)$ с помощью равносильных преобразований в тождественно истинную формулу $1$/

\textbf{Задача}. С помощью равносильных преобразований выяснить, является ли тождественно истинной формула $$\Phi = ((Y \Rightarrow Z) \land (X \Rightarrow V) \land (X \lor \lnot Z)) \Rightarrow (\lnot Y \lor 
V)$$

\underline{Решение}. Вспомним, что $X \lor \lnot X = 1$, $0 \Rightarrow X = 1$, $X \Leftrightarrow X = 1$.

$((Y \Rightarrow Z) \land (X \Rightarrow V) \land (X \lor \lnot Z)) \Rightarrow (\lnot Y \lor V) = \lnot((\lnot Y \lor Z) \land (\lnot X \lor V) \land (X \lor \lnot Z)) \lor (\lnot Y \lor V) = ((Y \land \lnot Z) \lor (X \land \lnot V) \lor (\lnot X \land Z)) \lor \lnot Y \lor V = ((Y \lor \lnot Z) \lor \lnot Y( \lor ((X \land \lnot V) \lor V) \lor (\lnot X \land Z) = ((Y \lor \lnot Y) \land (\lnot Z \lor \lnot Y)) \lor ((X \lor V) \land (\lnot V \lor V)) \lor (\lnot X \land Z) = \lnot Z \lor \lnot Y \lor V \lor (X \lor (\lnot X \land Z)) = \lnot Z \lor \lnot Y \lor V \lor ((X \lor \lnot X) \land (X \lor Z)) = \lnot Z \lor \lnot Y \lor V \lor X \lor Z = \ = (Z \lor \lnot Z) \lor \lnot Y \lor V \lor X = 1 \lor \lnot Y \lor V \lor X = 1$

\subsection{Алгоритм Квайна}
Алгоритм Квайна позволяет сократить полный перебор значений пропозициональных переменных за счёт последовательного фиксирования возможных значений 0 или 1 пропозициональных переменных и последующего анализа истинностных значений полученных формул с меньшим количеством переменных. Если по мере перебора комбинаций значений не находится ложное высказывание, очевидно, что формула является тождественно истинной.

При этом используются основные тавтологии и простейшие равенства.

\textbf{Задача}. С помощью алгоритма Квайна выяснить, является ли тождественно истинной формула $((Y \Rightarrow Z) \land (X \land V) \land (X \lor \lnot Z)) \Rightarrow (\lnot Y \lor V)$.

\underline{Решение}. Нетрудно догадаться, что перебирать значения стоит для переменной, встречающейся в формуле чаще остальных. В случае данной задачи все переменные встречаются одинаковое число раз.
\begin{enumerate}
    \item При фиксировании в исходной формуле $X = 1$ получаем $((Y \Rightarrow Z) \land (1 \Rightarrow V) \land (1 \lor \lnot Z)) \Rightarrow (\lnot Y \lor V)$, что равносильно $((Y \Rightarrow Z) \lor V) \Rightarrow (\lnot Y \lor V)$. Положим здесь $Y = 1$: $$((1 \Rightarrow Z) \land V) \Rightarrow (\lnot 1 \lor V)$$
    
    \dots

    Положим теперь $Y = 0$: $$((0 \Rightarrow Z) \lor V) \Rightarrow (\lnot 0 \lor V) = V \rightarrow 1 =  1$$

    \item При фиксировании в исходной формуле $X = 0$ получаем $((Y \Rightarrow Z) \land (0 \Rightarrow V) \land (0 \lor \lnot Z)) \Rightarrow (\lnot Y \lor V)$,  что равносильно  $((Y \Rightarrow Z) \land \lnot Z) \Rightarrow (\lnot Y \lor V)$.
    
    Положим здесь $Y = 1$: $((1 \Rightarrow Z) \land \lnot Z) \Rightarrow (\lnot 1 \lor V)$, что равносильно $(Z \lor \lnot Z) \Rightarrow V = 0 \Rightarrow V = 1$.

    Аналогично для $Y=0$.
\end{enumerate}

\subsection{Алгоритм редукции}
Алгоритм редукции используется при доказательстве тождественной истинности формул с большим количеством импликаций. Идея метода основывается на получении противоречия из предположения, что истинностное значение рассматриваемой формулы равно 0 при некоторых истиннстных значениях её пропозициональных переменных. При этом используется тот факт, что импликация ложна в том и только в том случае, если её посылка истинна и заключение ложно.