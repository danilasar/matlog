\chapter{15 марта. Логика предикатов}
\section{Логика предикатов}
кр будет проводиться во время лекции, предварительно 29 марта.

\dftion {\it Предикатом} называется утвержение, содержащее переменные $x_1,\dots,x_n$, которое превращается в высказывание при замене этих переменныз конкретными объектами из некоторой области возможных значений.

Обозначаются предикаты $P$, $Q$, $\dots$.

Переменные $x_1,\dots,x_n$ называются {\it предметными} или {\it индивидуальными переменными}. Число предметных переменных в предикате называется его {\it арностью} или {\it местностью}.

Более точно, предикат $P$ с $n$ предметныи переменными называется {\it $n$-арным} или {\it $n$-местным предикатом} и обозначается $P(x_1,\dots,x_n)$.

$M^n = \{(a_n, a_m): a_1, a_n \in M \}$

\dftion {\it Предикатом} называется утверждение, содержащее переменные $x_1, \dots, x_n$, которое превращается в высказывание при замене этих переменных конкретными объектами из некоторой области возможных значений $M$.

Истинностная функция предиката $F_P: M^n \to \{0,1\}$ определяется множеством истинности: $P^+ = \{(a_1, \dots, a_n) \in M^n: \lambda(P(a_1, \dots, a_n)) = 1\}$.

\dftion Предикат $P(x_1, \dots, x_n)$ на множестве M называется:
\begin{itemize}
    \item {\it тождественно истинным}, если $\forall i=\overline{1,n} \forall x_i = a_i \in M:$ высказывание $P(x_1, \dots, x_n)$ истинно, т. е. $P^+ = M^n$
    \item {\it тождественно ложным}, если $\forall i=\overline{1,n} \forall x_i = a_i \in M:$ высказывание $P(x_1, \dots, x_n)$ истинно, т. е. $P^+ = \varnothing$
    \item {\it выполнимым}, если $\exists x_1 = a_1 \in M, \dots, x_n = a_n \in M:$ высказывание $P(x_1, \dots, x_n)$ истинно, т. е. $P^+ \neq \varnothing$
    \item {\it опровержимым}, если $\exists x_1 = a_1 \in M, \dots, x_n = a_n \in M:$ высказывание $P(x_1, \dots, x_n)$ истинно, т. е. $P^+ \neq M^n$
\end{itemize}

\section{Алгебра предикатов}

{\it Отрицание $n$-местного предиката $P(x_1, \dots, x_n)$} определяетя как $n$-местный предикат $\neg P$, который при подстановке значений превращается в высказывание $\neg P(a_1, \dots, a_n)$, являющееся отрицанием высказывания $P(a_1, \dots, a_n)$.

{\it Конъюнокция $n$-местных предикатов $P(x_1, \dots, x_n)$ и $Q(x_1, \dots, x_n)$} определяется как $n$-местный предикат $P \land Q$, который при подстановке значений превращается в высказывание $P\land Q(a_1, \dots, a_n)$.

Для любого множества M допустимых значений переменных предикатов множества истинности предикатов взаимосвязаны с логическими операциями по следующим формулам: \\
$(\neg P)^+ =$ $M^n \backslash P^+$ \\
$(P \land Q)^+ =$ $P^+ \cap Q^+$ \\
$(P \lor Q)^+ =$ $P^+ \cup Q^+$ \\
$(P \Rightarrow Q)^+ =$ $(\neg P)^+ \cup Q^+$ \\
$(P \Leftrightarrow Q)^+ =$ $(P \Rightarrow Q)^+ \cap (Q \Rightarrow P)^+ =$ $(P^+ \cap Q^+) \cup ((\neg P)^+ \cap (\lnot Q)^+)$

{\it Примеры.}
\begin{enumerate}
    \item Пусть на множестве вещественных чисел $\mathbb R$ предикат $P(x)$ выражается неравенством $f(x) \leq 0$ и предикат $Q(x)$ выражается неравенством $g(x) \leq 0$. Тогда система неравенств $\begin{cases} f(x) \leq 0, \\ g(x) \leq 0 \end{cases}$ определяется как конъюнкция предикатов $P \land Q$ $\Rightarrow$ имеет множество решений $(P \land Q)^+ = P^+ \cap Q^+$, равное пересечению множеств решений неравенств системы.
    \item Пусть на множестве вещественных чисел $\mathbb R$ предикат $P(x)$ выражается неравенством $f(x) \leq 0$ и предикат $Q(x)$ выражается неравенством $g(x) \leq 0$. Тогда совокупность неравенств $\left[ \begin{gathered} f(x) \leq 0, \\ g(x) \leq 0 \end{gathered} \right.$ определяется как дизъюнкция предикатов $P \lor Q$ $\Rightarrow$ имеет множество решений $(P \lor Q)^+ = P^+ \cup Q^+$, равное объединению множеств решений неравенств системы.
\end{enumerate}


$\forall$ -- квантор общности (читается <<для всех>> --- от All), $\exists$ --- квантор существования (читается <<существует>> --- от Exist)

\dftion Результатом действия квантора общности $(\forall x_1)$ по переменной $x_1$ на $n$-местный предикат $P(x_1,\dots, x_n)$ называется $(n-1)$-местный предикат $(\forall x_1)P(x_1,x_2,\dots,x_n)$, который зависит от переменных $x_2,\dots,x_n$ и который при значениях $x_2=a_2,\dots,x_n=a_n$ в том и только том случае истинен на множестве $M$ допустимых значений переменной $x_1$, если при любых значениях $x_1 = a_1 \in M$ высказывание $P(a_1, a_2, \dots, a_n)$ истинно. \\
$(\forall x_1)P(x_1,x_2,\dots,x_n)$ $\overset{df}\Leftrightarrow$ при любых значениях $x_1 = a_1 \in M$ высказывание $P(a_1, a_2, \dots, a_n)$ истинно. \\
$(\forall x_1)P(x_1,x_2,\dots,x_n)$ -- предикат от переменных $x_2,\dots,x_n$ \\
при $x_2 = a_2,\dots,x_n = a_n$ истеннен на $M$ $\Leftrightarrow$ предикат $P(x_1, a_2, \dots, a_n)$ тождественно истинен на M.


\dftion Результатом действия квантора существования $(\exists x_1)$ по переменной $x_1$ на $n$-местный предикат $P(x_1,\dots, x_n)$ называется $(n-1)$-местный предикат $(\exists x_1)P(x_1,x_2,\dots,x_n)$, который зависит от переменных $x_2,\dots,x_n$ и который при значениях $x_2=a_2,\dots,x_n=a_n$ в том и только том случае истинен на множестве $M$ допустимых значений переменной $x_1$, если при некотором значении $x_1 = a_1 \in M$ высказывание $P(a_1, a_2, \dots, a_n)$ истинно. \\
$(\exists x_1)P(x_1,x_2,\dots,x_n)$ $\overset{df}\Leftrightarrow$ при хотя бы одном значении $x_1 = a_1 \in M$ высказывание $P(a_1, a_2, \dots, a_n)$ истинно.

\underline{Пример}. \\
$\underset{P_1(\epsilon)}{\underbrace{(\forall \epsilon > 0)}} \underset{P_2(\delta)}{\underbrace{(\exists \delta > 0)}}$ \\
$(\exists Q(x))P(x) \overset{df}= (\exists x)(Q(x) \land P(x))$
$(\forall Q(x))P(x) \overset{df}= (\forall x)(Q(x) \underset{\bcancel{\cancel{\land}}}\Rightarrow P(x))$


Другие кванторы, как правило, являются сокращениями формул.


\dftion \textbf{Квантор существования и единственности}:

$(\exists ! x)P(x) = (\exists x)(P(x) \land ((\forall y)(P(y) \Rightarrow x = y)))$

\dftion \textbf{Ограниченный квантор существования} $\Big(\exists Q(x)\Big)$ (читается <<существует $x$, удовлетворяющий $Q(x)$, для которого выполняется $P(x)$>>):

$\Big(\exists Q(x)\Big)P(x) = (\exists x)(Q(x) \land P(x))$

\dftion \textbf{Ограниченный квантор общности} $\Big(\forall Q(x)\Big)$ (<<для всех $x$, удовлетворяющих $Q(x)$, выполняется $P(x)$>>):

$\Big(\forall Q(x)\Big)P(x) = (\forall x)(Q(x) \then P(x))$

\dftion \textbf{Алгебра предикатов} --- множество всех предикатов $\mathscr{P}$ с логическими операциями $\lnot, \land, \lor, \then, \eq$ и операциями квантификации $(\forall x), (\exists x)$ для всех предметных переменных $x$.

\subsection{Формулы алгебры предикатов}

Свойства алгебры предикатов $\mathscr{P}$ описываются с помощью специальных формул, которые строятся из символов предикатов и предметных переменных с помощью специальных вспомогательных символов --- скобок и знаков логических операций.

\dftion \textbf{Алфавит} алгебры предикатов состоит из следующих символов:
\begin{enumerate}
    \item {\it предметные переменные $x_1,x_2,\dots$}, которые используются для обозначения элеметнов множества допустимых значений
    \item $n$-местные {\it предикатные символы $P,Q,\dots$}, которые используются для обозначения $n$-местных предикатов на множестве допустимых значений
    \item символы логических операций $\neg, \land, \lor, \Rightarrow, \Leftrightarrow, \forall, \exists$
    \item вспомогательные символы (скобки, запятая и другие)
\end{enumerate}
{\renewcommand{\arraystretch}{1.5}
\setlength{\tabcolsep}{5pt}
\rowcolors{3}{black!10!white!50}{black!2!white!90}
    \begin{longtable}[h!]{ |c|c| }
        \hline
        Вспом. выс. & Лог. пред \\
        \hline
        \endhead
        $X$ & $P(x_1, \dots, x_n)$ \\
        \hline
    \end{longtable}}


\dftion \textbf{Формулы} алгебры предикатов определяются по индукции следующим образом:
\begin{enumerate}
    \item для любого $n$-местного предикатного символа $P$ и любых $n$ предметных переменных $x_1, \dots, x_n$ выражение $P(x_1, \dots, x_n)$ есть формула, которая называется {\it элементарной} (или {\it атомарной}) {\it формулой};
    \item если $\Phi, \Psi$ -- формулы, то формулами являются также выражения: $(\neg \Phi), (\Phi \land \Psi), (\Phi \lor \Psi), (\Phi \Rightarrow \Psi), (\Phi \Leftrightarrow \Psi)$
    \item если $\Phi$ - формула и $x$ - предметная переменная, то формулами являются также выражения $(\forall x)\Phi$, $(\exists x)\Phi$; при этом переменная $x$ и формула $\Phi$ называется {\it областью действия} соответствующего квантора.
\end{enumerate}
Приоритеты: кванторы, отрицание, конъюнкция, дизъюнкция и остальные. \\
{\it Пример. } $\underset{3}{\underline{\underset{2}{\underline{\underset{1}{\underline{(\forall x)}}P(x)}} \underset{\text{\it не зависит от } \forall}{\land Q(x)}}}$

Если в формулу $\Phi$ входят переменные $x_1, \dots, x_n$, то записывают $\Phi = \Phi(x_1, \dots, x_n)$.

Вхождение предметной переенной $x$ в формулу $\Phi$ называется {\it связным}, если она находится в области действия одного из кванторов по этой переменной. В противном случае вхождение называется {\it свободным}.

Формула без свободных вхожений переменных называется \textit{замкнутой формой} или \textit{предложением}.

Фактически формула определяет предикат с переменными, которые входят в формулу свободно.

\subsection{Интерпретации формул алгебры предикатов}

{\it Область интерпретации} -- непустое множество $M$, которое является областью возможных значений всех предметных переменных.

$n$-местным предикатным символам $P$ присваиваются конкретные значения $P_M$ $n$-местных предикатов на множестве $M$.

Соответствие $\beta: P \to P_M$ называется {\it интерпретацией предикатных символов}.

Область интерпретации $M$ вместе с интерпретацией предикатных символов $\beta$ называется {\it интерпретацией формул алгебры предикатов} и обозначается $(M, \beta)$ или просто $M$. \\
{\it Пример. }$M \neq \varnothing$ \\

При наличии интерпретации $M$ конкретные значения предметным переменным формул алгебры предикатов присваиваются с помощью отображения $\alpha$ множества всех предметных переменных $X$ в область интерпретации $M$. Такие интерпретации называются {\it оценками} предметных переменных.

\dftion \textbf{Выполнимость формулы} $\Phi$ в интерпретации $M$ при оценке $\alpha$ обозначается $M \vDash_\alpha \Phi$ -- читается "формула $\Phi$ истинна в интерпретации M при оценке $\alpha$" и определяется следующим образом:
\begin{enumerate}
    \item если $\Phi = P(x_1, \dots, x_n)$ для $n$-местного предикатного символа $P$ и предметных переменных $x_1, \dots, x_n$, то $M \vDash_\alpha \Phi$ тогда и только тогда, когда высказывание $P_M(\alpha(x_1),\dots,\alpha(x_n))$ истинно;
    \item если $\Phi = \neg \Psi$ для формулы $\Psi$, то $M \vDash_\alpha \Phi$ $\Leftrightarrow$ неверно, что $M \vDash_\alpha \Psi$;
    \item если $\Phi = \Phi_1 \land \Phi_2$ для формул $\Phi_1, \Phi_2$, то $M \vDash_\alpha \Phi$ $\Leftrightarrow$ $M \vDash_\alpha \Phi_1$ и $M \vDash_\alpha \Phi_2$
    \item если $\Phi = \Phi_1 \lor \Phi_2$ для формул $\Phi_1, \Phi_2$, то $M \vDash_\alpha \Phi$ $\Leftrightarrow$ $M \vDash_\alpha \Phi_1$ или $M \vDash_\alpha \Phi_2$
    \item если $\Phi = \Phi_1 \lor \Phi_2$ для формул $\Phi_1, \Phi_2$, то $M \vDash_\alpha \Phi$ $\Leftrightarrow$ неверно, что $M \vDash_\alpha \Phi_1$ и $M \vDash_\alpha \neg \Phi_2$
    \item если $\Phi = \Phi_1 \eq \Phi_2$ для формул $\Phi_1, \Phi_2$, то $M \vDash_\alpha \Phi$ тогда и только тогда, когда $M \vDash_\alpha \Phi_1$ и $M \vDash_\alpha \lnot \Phi_2$;
    \item если $\Phi = (\forall x)\Psi$ для некоторой формулы $\Psi$, то $M \vDash_\alpha \Phi$ тогда и только тогда, когда $M \vDash_{\alpha'} \Psi$ для всех оценок $\alpha'$, отличающихся от оценки $\alpha$ возможно только на элементе $x$;
    \item если $\Phi = (\exists x)\Psi$ для некоторой формулы $\Psi$, то $M \vDash_\alpha \Phi$ тогда и только тогда, когда $M \vDash_{\alpha'}$ для некоторой оценки $\alpha'$, отличающейся от оценки $\alpha$ возможно только на элементе $x$.
\end{enumerate}