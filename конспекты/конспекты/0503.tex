\chapter{3 мая. Аксиоматический метод}
Было определено множество формул алгебры высказываний $F_{AB}$.

Затем было выделено подмножество формул $T_{AB} \subset F_{AB}$, состоящее из специальных формул --- тавтологий.

При этом в основе определения тавтологии лежит понятие интерпретации формул, то есть придание некоторого конкретного содержательного смысла входящих в них переменных. Такой подход к логическим формулам носит теоретико-множественный характер и называется \textit{семантическим}.

Альтернативой семантического подхода является \textit{синтаксический подход}, при котором логические формулы выводятся из первоначально выделенного множества формул --- аксиом --- по определённым правилам преобразования формул логического языка без привлечения вспомогательных теоретико-множественных понятий.

Построение математических теорий в виде аксиоматических теорий соответствующих формальных исчислений составляет суть \textit{аксиоматического метода в математике}.

Простейшей аксиоматической теорией является \textit{аксиоматическая логика высказываний}, которая строится на основе соответствующего формального исчисления, называемого \textit{исчислением высказываний} (сокращённо ИВ).

\section{Исчисление высказываний}
Множество аксиом Ax(ИВ) исчисления высказываний описывается следующими тремая \textit{схемами аксиом}:
\begin{enumerate}
    \item $(A_1)$ $(\Phi \then (\Psi \then \Phi))$,
    \item $(A_2)$ $((\Phi_1 \then (\Phi_2 \then \Phi_3)) \then ((\Phi_1 \then \Phi_2) \then (\Phi_1 \then \Phi_3)))$,
    \item $(A_3)$ $((\lnot \Phi \then \lnot \Psi)\then ((\lnot \Phi \then \Psi)\then \Phi))$,
\end{enumerate}
где $\Phi, \Psi, \Phi_i$ ($i = \overline{1,3}$) --- произвольные формулы исчисления высказываний.

Исчисление высказываний имеет \textbf{правило вывода}, которе называется \textbf{правилом заключения} или правилом \textit{modus ponens} (сокращённо MP) и которое для произвольных формул исчисления высказываний $\Phi, \Psi$ определяется по формуле MP $(\Phi \then \Psi, \Phi) = \Psi$.

Символически это правило вывода записывается следующей схемой:
\begin{equation*}
    MP: \begin{matrix}
        \Phi \then \Psi, \Phi \\
        \Psi
    \end{matrix}.
\end{equation*}

В основе алгоритма вывода \textbf{теорем} исчисления высказываний лежит следующее понятие.

\dftion Формула $\Phi$ называется \textit{теоремой исчисления высказываний}, если найдётся такая конечная последовательность формул $\Phi_1, \dots, \Phi_n$, в которой:
\begin{enumerate}
    \item $\Phi_n = \Phi$
    \item каждая формула $\Phi_i$ либо являтся аксиомой, либо получается из некоторых двух предыдущих формул $\Phi_i, \Phi_k$ ($1 \leq j, k < i$) по правилу вывода MP.
\end{enumerate}

Последовательность формул $\Phi_1, \dots, \Phi_n$ называется \textbf{выводом} или \textbf{доказательством} формулы $\Phi$.

\dftion Вывод формулы $\Phi$ сокращённо обозначают символом $\vdash \Phi$ и говорят, что $\Phi$ есть теорема. Множество всех таких теорем обозночается символом Th(ИВ) и называется \textbf{теорией исчисления высказываний}.

Главной целью построения исчисления высказываний является определение такой теории Th(ИВ), которая совпадает с множеством тавтологий $T_{AB}$.

\textbf{Лемма}.
Справедливы следующие утверждения:
\begin{enumerate}
    \item всякая аксиома ИВ является тавтологией;
    \item результат применения правила вывода MP к люым тавтологиям $\Phi \then \Psi, \Phi$ даёт тавтологию $\Psi$;
    \item всякая теорема ИВ является тавтологией, то есть выполняется $Th(\text{ИВ}) \supset T_{\text{АВ}}$.
\end{enumerate}

\textbf{Теорема полноты ИВ}.
Всякая тавтология является теоремой ИВ, т. е. выполняется $T_\text{АВ} \subset Th(\text{ИВ})$ и, следовательно, $T_{AB}=Th(\text{ИВ})$.

Следствия теоремы полноты ИВ.

\textbf{Теорема о непротиворечивости ИВ}.

В исчислении высказываний невозможно доказать никакую формулу $\Phi$ вместе с её отрицанием $\lnot \Phi$.



\textbf{Теорема о разрешимости ИВ}.
Существует универсальная эффективная процедура (алгоритм), которая для любой формулы определяет, является ли эта формула теоремой ИВ.

\section{Исчисление предикатов}
Множество аксиом Ax(ИП) исчисления предикатов описываетя пятью \textit{схемами аксиом} --- тремя определёнными в предыдущем разделе схемами схемами $(A_1)-(A_3)$, в которых $\Phi, \Psi, \Phi_i (i = \overline{1,3})$ являются произвольными формулами исчисления предикатов, и двумя новыми схемами:
\begin{enumerate}
    \setcounter{enumi}{3}
    \item $(A_4)$ $(\forall x)\Phi(x) \then \Phi(y)$ для произвольной формулы $\Phi(x)$, в которую $y$ не входит связно;
    \item $(A_5)$ $(\forall x)(\Phi \then \Psi(x)) \then (\Phi \then (\forall x)\Psi(x))$ для таких формул $\Phi, \Psi$, что $x$ в формулу $\Phi$ не входит свободно.
\end{enumerate}

Исчисление предикатов имеет два \textbf{правила вывода} --- правило modus ponens (сокращнно MP) и правило обобщения (сокращённо Gen), которые для произвольных формул исчисления предикатов $\Phi, \Psi$ символически записываются следующими схемами:
\begin{figure}[H]
    \centering
    \begin{tabular*}{0.5\textwidth}{@{\extracolsep{\fill}}ccc@{}}
        $MP: \begin{matrix}
            \Phi \then \Psi, \Phi \\
            \Psi
        \end{matrix}$ &
        и &
        $Gen: \begin{matrix}
            \Phi \\
            (\forall x)\Phi
        \end{matrix}$.
    \end{tabular*}
\end{figure}

\dftion Формула $\Phi$ называется \textbf{теоремой исчисления предикатов}, если найдётся такая последовательность $\xses[\Phi]$, в которой $\Phi_n = \Phi$ и каждая формула $\Phi_i$ либо является аксиомой, либо получается из некоторых предыдущих формул этой последовательности по одному из правил вывода MP или Gen. При этом $\Phi_1, \dots, \Phi_n$ называется \textbf{выводом} или \textbf{доказательством} формулы $\Phi$.

Вывод формулы $\Phi$ обозначают $\vdash \Phi$ и говорят, что $\Phi$ есть теорема. Множество всех таких теорем обозначается символом Th(ИП) и называется \textit{теорией исчисления предикатов}.

Цель построения исчисления предикатов --- определение такой теории Th(ИП), которая совпадает со множеством тавтологий $T_{\text{АП}}$.

\textbf{Лемма 1}.
Справедливы следующие утверждения:
\begin{enumerate}
    \item всякая аксиома ИП является тавтологией;
    \item результат применения правил вывода MP и Gen к тавтологиям является тавтологией;
    \item любая теорема ИП является тавтологией ИП, т. е. имеет место включение $Th(\text{ИП}) \supset T_{\text{АП}}$.
\end{enumerate}

Доказательство $T_{\text{АП}} \subset Th(\text{ИП})$ был получено астрийским математиком К. Геделем в 1930 году.

\textbf{Теорема полноты ИП}.

Формула исчисления предикатов в том и только в том случае является тавтологией, если она есть теорема ИП, т. е. выполняется равенство $T_{\text{АП}}=Th(\text{ИП})$.

Таким образом, ИП является адекватным инструментом получения логических законов.

\textbf{Теорема о непротиворечивости ИП}
В исчислении предикатов невозможно доказать никакую формулу $\Phi$ вместе с её отрицанием $\lnot \Phi$.

С другой стороны, английский математик А. Черч в 1936 году доказал следующий принципиально важный результат.

\textbf{Теорема о неразрешимости ИП}.

Не существует универсальной эффективной процедуры (алгоритма), которая для любой формулы определяет, является ли эта формула теоремой ИП.

\section{Элементы теории алгоритмов}
Важные математические проблемы имеют вид:

для некоторого данного множества $X$ найти эффективную процедуру (т. е. алгоритм), с помощью которой можно для каждого элемента $x$ этого множества $X$ определить за конечное число шагов, будет этот элемент обладать некоторым данным свойством $P$ или нет (т. е. $x \in P^+$ или $x \not\in P^+$).

\textit{Решение} такой проблемы --- построение и обоснование искомого алгоритма.

\dftion \textbf{Массовые задачи} --- задачи распознавания и оптимизации.

Примеры массовых задач:
\begin{itemize}
    \item ВЫП (SAT) --- задача выполнимости формулы логики высказываний
    \item ТЕОРЕМА (THM) --- задача доказуемости формулы логики предикатов
\end{itemize}