\section{Тема 1. Алгебра высказываний}
\subsection{Понятие высказывания}
\dftion \textit{Высказывание} --- повествовательное предложение, о котором можно судить, истинное оно или ложное. Обозначается заглавными, обычно первыми, символами латинского алфавита: $A, B, C$\dots
\subsection{Логические операции над высказываниями}
Алгебра высказываний задаётся операциями отрицания ($\lnot$), конъюнкции ($\land$), дизъюнкции ($\lor$), следования ($\then$) и равносильности ($\eq$) над множеством $P$:
\begin{equation}
    \mathscr{P} = <P, \lnot, \land, \lor, \then, \eq>
\end{equation}

\begin{itemize}
    \item $\lnot A$ истинно тогда и только тогда, когда $A$ ложно;
    \item $A \land B$ истинно тогда и только тогда, когда $A$ и $B$ одновременно истинны;
    \item $A \lor B$ истинно тогда и только тогда, когда хотя бы одно из значений истинно;
    \item $A \then B$ ложно тогда и только тогда, когда $A$ истинно, а $B$ ложно;
    \item $A \eq B$ истинно тогда и только тогда, когда $A$ и $B$ истинны или ложны одновременно
\end{itemize}
\subsection{Формулы}
\dftion \textbf{Пропозициональные формулы} --- это формулы, с помощью котороых описываются свойства алгебры высказываний $\mathscr{P}$. \textbf{Пропозициональные связки} --- это символы логических операций в них, а \textbf{пропозициональные переменные} --- это символы $X, Y, Z, \dots$, которые используются для обозначения высказываний.

\dftion Формулы алгебры логики образуют множество $\mathfrak I$ и индуктивно определяются по правилам:
\begin{enumerate}
    \item Каждая пропозициональная переменная является формулой
    \item Если $\Phi, \Psi$ --- формулы, то формулами являются и выражения операций над ними.
\end{enumerate}

\dftion Приоритет логических операций по убыванию: отрицание, конъюнкция, дизъюнкция, а также следование и равносильность, имеющие равный приоритет.

\textbf{Лемма об основных равенствах формул.} \label{q1:lemma}
2 безымянных, следование, равносильность, сочетательный/перестановочный закон, де Морган, поглощение и двойное отрицание (всего 9):
\begin{enumerate}
    \item $A \lor (B \lor C) = (A \lor B) \lor C$ \Big($A \land (B \land C) = (A \land B) \land C$\Big) --- сочетательный закон;
    \item $A \lor B = B \lor A$ \Big($A \land B = B \land A$\Big) --- перестановочный закон;
    \item $A \lor (B \land C) = (A \lor B) \land (A \lor C)$ \Big($A \land (B \lor C) = (A \land B) \lor (A \land C)$\Big);
    \item $X \land X = X$ \Big($X \lor X = X$\Big);
    \item $\lnot(A \land B) = \lnot A \lor \lnot B$ \Big($\lnot(A \lor B) = \lnot A \land \lnot B$\Big) --- закон де Моргана;
    \item $X \lor (X \land Y) = X$ \Big($X \land (X \lor Y) = X$\Big) --- закон поглощения;
    \item $\lnot \lnot X = X$ --- закон двойного отрицания;
    \item $X \then Y = \lnot X \lor Y = \lnot (X \land \lnot Y)$
    \item $X \eq Y = (X \land Y) \lor (\lnot X \land \lnot Y) = (X \lor \lnot Y) \land (\lnot X \lor Y)$
\end{enumerate}

\subsection{Истинностные значения формул}
\dftion \textbf{Истинностное значение} $\lambda(A)$ ставит в соответствие высказыванию 1, если оно истинно, или 0, если оно ложно.

\dftion Если в формулу $\Phi$ входят переменные $X_1, \dots, X_n$, то записывают $\Phi = \Phi(X_1, \dots, X_n)$. Формула определяет функцию $n$ переменных $F$, которая каждому упорядоченному набору $(\lambda(X_1), \dots, \lambda(X_n))$ из $n$ элементов множества $\{0, 1\}$ ставит в соответствие $\lambda(\Phi(X_1, \dots, X_n))$ этого же множества.
\subsection{Тавтологии}
\dftion Формула $\Phi$ называется
\begin{itemize}
    \item \textbf{тавтологией} (тождественно истинной), если она истинна при любых значениях переменных, и обозначается $\vDash \Phi$;
    \item \textit{выполнимой}, если найдутся такие значения переменных, при которых $\Phi$ является истинной;
    \item \textit{опровержимой}, если найдутся такие значения переменных, при которых $\Phi$ является ложной;
    \item \textbf{противоречием} (тождественно ложной), если она ложна (или противоречива) при любых значениях переменных.
\end{itemize}
\dftion Тавтологии являтся общими схемами построения логических высказываний и выражают \textbf{логические законы}.
\subsection{Методы доказательства тавтологий}
\begin{enumerate}
    \item Прямой метод
    \item Алгебраический метод
    \item Метод Квайна
    \item Метод редукции
    \item Метод семантических таблиц
    \item Метод резолюций
\end{enumerate}

\subsubsection{Прямой метод}
Прямой метод подразумевает перебор всех возможных значений переменных и построение таблицы истинности.

\subsubsection{Алгебраический метод}
Предполагает приведение формулы $\Phi$ к тождественно истинной формуле $1$ путём равносильных преобразований.

\subsubsection{Метод Квайна}
Предполагает построение дерева решений с подстановкой возможных значений очередной переменной на каждой очередной высоте дерева.

\subsubsection{Метод редукции}
Основывается на движении от обратного: истинностное значение формулы $\Phi$ принимается за 0 и, на основании этого, восстанавливаются значения исходных переменных. Если возникает противоречие, формула является тождественно истинной. Особенно удобен в формулах с большим количеством импликаций благодаря её свойству: она ложна тогда и только тогда, когда посылка истинна, а заключение ложно.

\subsubsection{Метод резолюций}
\dftion Пусть для некоторой переменной $X$ дизъюнкты представлены в виде $D_1 = D_1' \lor X, D_2 = D_2' \lor \lnot X$. Тогда $Res_X(D_1, D_2) = D_1' \lor D_2'$ является \textbf{резольвентой} дизъюнктов $D_1, D_2$ по переменной $X$. По определению $Res(X, \lnot X) = 0$.
\underline{Свойство}. $D_1, D_2$ выполнимы $\then$ $Res(D_1, D_2)$ выполнима
\dftion \textbf{Резолютивный вывод} формулы $\Phi$ из множества дизъюнктов $S = \{D_1, \dots, D_n\}$ --- последовательность формул, соответствующая двум правилам:
\begin{itemize}
    \item $\Phi_n = \Phi$,
    \item $\Phi_i = Res(\Phi_j, \Phi_k)$, где $1 \leq j, k < i$
\end{itemize}
\textbf{Основная теорема метода резолюций}. Множество дизъюнктов $S = \{D_1, \dots, D_n\}$ противоречиво $\eq$ $\exists$ резолютивный вывод значения 0 из множества $S$.