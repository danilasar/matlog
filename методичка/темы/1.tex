\section{Тема 1. Алгебра высказываний}
\subsection{Понятие высказывания}
\dftion \textit{Высказывание} --- повествовательное предложение, о котором можно судить, истинное оно или ложное. Обозначается заглавными, обычно первыми, символами латинского алфавита: $A, B, C$\dots
\subsection{Логические операции над высказываниями}
Алгебра высказываний задаётся операциями отрицания ($\lnot$), конъюнкции ($\land$), дизъюнкции ($\lor$), следования ($\then$) и равносильности ($\eq$) над множеством $P$:
\begin{equation}
    \mathscr{P} = <P, \lnot, \land, \lor, \then, \eq>
\end{equation}

\begin{itemize}
    \item $\lnot A$ истинно тогда и только тогда, когда $A$ ложно;
    \item $A \land B$ истинно тогда и только тогда, когда $A$ и $B$ одновременно истинны;
    \item $A \lor B$ истинно тогда и только тогда, когда хотя бы одно из значений истинно;
    \item $A \then B$ ложно тогда и только тогда, когда $A$ истинно, а $B$ ложно;
    \item $A \eq B$ истинно тогда и только тогда, когда $A$ и $B$ истинны или ложны одновременно
\end{itemize}
\subsection{Формулы}
2 безымянных, следование, равносильность, сочетательный/перестановочный закон, де Морган, поглощение и двойное отрицание (всего 9):
\begin{enumerate}
    \item $A \lor (B \lor C) = (A \lor B) \lor C$ \Big($A \land (B \land C) = (A \land B) \land C$\Big) --- сочетательный закон;
    \item $A \lor B = B \lor A$ \Big($A \land B = B \land A$\Big) --- перестановочный закон;
    \item $A \lor (B \land C) = (A \lor B) \land (A \lor C)$ \Big($A \land (B \lor C) = (A \land B) \lor (A \land C)$\Big);
    \item $X \land X = X$ \Big($X \lor X = X$\Big);
    \item $\lnot(A \land B) = \lnot A \lor \lnot B$ \Big($\lnot(A \lor B) = \lnot A \land \lnot B$\Big) --- закон де Моргана;
    \item $X \lor (X \land Y) = X$ \Big($X \land (X \lor Y) = X$\Big) --- закон поглощения;
    \item $\lnot \lnot X = X$ --- закон двойного отрицания;
    \item $X \then Y = \lnot X \lor Y = \lnot (X \land \lnot Y)$
    \item $X \eq Y = (X \land Y) \lor (\lnot X \land \lnot Y) = (X \lor \lnot Y) \land (\lnot X \lor Y)$
\end{enumerate}
\subsection{Истинностные значения формул}
\dftion \textit{Истинностное значение }
\subsection{Тавтологии}
\subsection{Методы доказательства тавтологий}