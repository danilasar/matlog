\section{Элементы теории алгоритмов}
\subsection{Понятие алгоритма и основные математические модели алгоритма}
\dftion \textbf{Алгоритм} --- совокупность инструкций, которе дают решение некоторой массовой задачи, обладающая четырьмя свойствами:
\begin{enumerate}
    \item дискретность
    \item детерменированность
    \item элементарность шагов
    \item массовость
\end{enumerate}

\dftion \textbf{Модели алгоритма} --- основные варианты математического определения алгоритма:
\begin{enumerate}
    \item рекурсивная функция
    \item машина Тьюринга
    \item нормальный алгорифм
    \item формальная грамматика
\end{enumerate}
\subsection{Машина Тьюринга и вычисляемая ею функция}
Машина Тьюринга схематически определяется следующим образом:
\begin{enumerate}
    \item Символы внешнего алфавита $\Sigma = \{0, 1, \dots\}$ записываются в ячейки конечной ленты, которая называется \textbf{внешней памятью машины}, при необходимости в концы дописывается \textit{пустой} символ *
    \item Символы внутреннего алфавита $Q = \{q_S, q_F, \dots\}$ обозначают состояния \textbf{управляющего устройства машины} (УУ) с \textbf{просматривающей головкой}, которая может перемещаться вдоль ленты и в каждый момент времени $t$ просматривать одну ячейку
    \item \textbf{Программа машины} $$\Pi = \{T(q, a : q \in Q \backslash \{q_F\}, a \in \Sigma)\}$$ состоит из \textbf{команд} $T: q, a \to q' a' X$, которые в зависимости от состояния машины $q$ и символа $a$ в просматривающей головки изменяют состояние $q$ на $q'$, символ $a$ на $a'$ и сдвигают просматривающую головку в зависимости от значения $X = \{R, L, S\}$ вправо, влево или оставляют на месте.
    \item Машина начинает работать в \textbf{начальном состоянии} $q_S$ и завершает работу в \textbf{заключительном состоянии} $q_F$
\end{enumerate}

\dftion \textbf{Конфигурация} машины Тьюринга описывается \textbf{машинным словом} $M = \alpha q a \beta$. Она является \textit{начальной} или \textit{конечной}, если в ней содержится символ соответственно начального или заключительного состояния.