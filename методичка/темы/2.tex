\section{Логическая равносильность}
\subsection{Равносильность формул}
\dftion \textbf{Логическая равносильность} обозначается символом $=$ ($\cong$) и означает, что формулы $\Phi, \Psi$ принимают одинаковые значения при любых истинностных значениях переменных, то есть выполняется следование $\vDash \Phi \eq \Psi$.
\subsection{Равносильные преобразования формул}
См. \hyperref[q1:lemma]{лемму об основных равенставх формул}.
\subsection{Нормальные формы для формул алгебры высказываний}
\dftion \textbf{Нормальной формулой} называется формула, содержащая только операции $\lnot$, $\land$ и $\lor$. По лемме об основных равенствах формул любую формулу можно привести к нормальной.

\dftion \textbf{Литера} --- пропозиционная переменная $X$ или её отрицание$\lnot X$. Обозначается $X^1=X$ и $X^0=\lnot X$.

\dftion \textbf{Конъюнкт} (\textbf{дизъюнкт}) --- конъюнкция (дизъюнкция) литер или одна литера.

\dftion \textbf{Конъюнктивная нормальная форма} (КНФ) --- конъюнкция дизъюнктов или один дизъюнкт.


\dftion \textbf{Дизъюнктивная нормальная форма} (ДНФ) --- дизъюнкция конъюнктов или один конъюнкт.

КНФ (ДНФ) называется совершенной, если в ней содержатся все её конъюнкты (дизъюнкты) содержат все пропозициональные переменные.

\textbf{Теорема}. Людая формула равносильна некоторой КНФ и некоторой ДНФ.

\underline{Алгоритм получения КНФ (ДНФ)}.
\begin{enumerate}
    \item Преобразовываем равносильности и следования в конъюнкции и дизъюнкции
    \item Согласно законам де Моргана меняем формулу так, чтобы занести все отрицания в скобки, и сокращаем двойные отрицания
    \item Согласно законам дистрибутивности приводим формулу непосредственно к КНФ (ДНФ)
\end{enumerate}

\underline{Алгоритм получения СКНФ (СДНФ)} формулы $\Phi = \Phi(X_1, \dots, X_n)$:
\begin{enumerate}
    \item Строим таблицу истинности и добавляем два стлобца --- для совершенных конъюнктов и для совершенных дизъюнктов
    \item Там, где истинностное значение формулы равно 1, в совершенные конъюнкты записываем $X_1^{\alpha_1} \land \dots \lor X_n^{\alpha_n}$, где $\alpha_i = \lambda(X_i)$, а в совершенные дизъюнкты ставим прочерк
    \item Там, где формула ложна, в совершенные дизъюнкты записываем $X_1^{\alpha_1} \lor \dots \lor X_n^{\alpha_n}$, где $\alpha_i = 1-\lambda(X_i)$, а в совершенные конъюнкты ставим прочерк
    \item СКНФ (СДНФ) будет конъюнкцией совершенных дизъюнктов (конъюнктов)
\end{enumerate}
