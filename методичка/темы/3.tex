\section{Логическое следование формул}
\dftion \textbf{Логическое следование} обозначается символом $\vDash$ и означает, что при подстановке в некоторые формулы $\Phi_1, \dots, \Phi_n$ вместо их переменных конкретных значений $A_1, \dots, A_n$ из истинности $\Phi_1(A_1,\dots,A_n),\dots,\Phi_n(A_1,\dots,A_n)$ следует истинность высказывания $\Phi(A_1,\dots,A_n)$. $\Phi$ при этом называется \textbf{следствием}, а $\Phi_1, \dots, \Phi_n$ --- \textbf{посылками}.

Частными случаями следования является:
\begin{itemize}
    \item $\vDash \Phi$ или $1 \vDash \Phi$ --- тавтология;
    \item $\Phi \vDash$ или $\Phi \vDash 0$ --- противоречие
\end{itemize}

Основные правила логического следования:
\begin{itemize}
    \item Правило отделения: $\Phi, \Phi \then \Psi \vDash \Psi$;
    \item Правило контрапозиции: $\Phi \then \Psi \vDash \lnot \Psi \then \lnot \Phi$;
    \item Правило цепного следования: $\Phi \then \Psi, \Psi \then \Xi \vDash \Phi \then \Xi$;
    \item Правило перестановки посылок: $\Phi \then (\Psi \then \Xi) \vDash \Psi \then (\Phi \then \Xi)$
\end{itemize}

\subsection{Методы доказательсва логического следования формул}
\begin{enumerate}
    \item \textbf{Прямой метод}
    \item \textbf{Алгебраический метод} предполагает вывод тождественно истинного высказывания 1 путём равносильных преобразований
    \item \textbf{Метод Квайна} предполагает построение дерева путём подстановки значений из множества $\{0,1\}$ вместо очередной переменной на очередной единице глубины дерева.
    \item \textbf{Метод редукции} предполагает принятие высказывания за ложное и доказательства противоречивости путём пошагового восстановления исходных значений переменных
    \item \textbf{Метод семантических таблиц}
    \item \textbf{Метод резолюций} предполагает построение резолютивного вывода нуля из КНФ
\end{enumerate}

\subsection{Метод резолюций в логике высказываний}
\dftion Пусть для некоторой переменной $X$ дизъюнкты $D_1, D_2$ представимы в виде $D_1 = D_1' \lor X$, $D_2 = D_2' \lor \lnot X$. Тогда $Res_X(D_1, D_2) = D_1' \lor D_2'$ --- \textbf{резольвента дизъюнкктов} $D_1, D_2$ по переменной $X$. По определению, $Res(X, \lnot X) = 0$

\underline{Свойство}. Если $D_1 = D_1' \lor X, D_2 = D_2' \lor \lnot X$ выполнимы, то выполнима и $Res_X(D_1, D_2)$.

\dftion Резолютивным выводом формулы $\Phi$ из множества дизъюнктов $S = \{D_1, \dots, D_m\}$ называется такая последовательность формул $\Phi_1, \dots, \Phi_n$, что:
\begin{enumerate}
    \item $\Phi_n = \Phi$;
    \item каждая из формул $\Phi_i (i = 1,\dots,n)$ либо принадлежит множеству $S$, либо является резольвентой $\Phi_i = Res(\Phi_j, \Phi_k)$ предыдущих формул $\Phi_j, \Phi_k$ при некоторых $1 \leq j, k < i$.
\end{enumerate}

\subsection{Основная теорема метода резолюций в логике высказываний --- теорема полноты резолютивного вывода}


\textbf{Основная теорема метода резолюций}. Множество дизъюнкктов $S=\{D_1,\dots,D_m\}$ противоречиво в том и только в том случае, если существует резолютивный вывод значения 0 из множества $S$.

Логическое следование $\Phi_1, \dots, \Phi_n \vDash \Phi$ равносильно существованию резолютивного вывода значения 0 из множества дизъюнктов $S = \{D_1, \dots, D_m\}$ конъюнктивной нормальной формы формулы $\Phi_1 \land \dots \land \Phi_n \land \lnot \Phi$.

\underline{Алгоритм проверки логического следования формул} $\Phi_1, \dots, \Phi_n \vDash \Phi$:
\begin{enumerate}
    \item Составить формулу $$\Psi = \Phi_1 \land \dots \land \Phi_n \land \lnot \Phi$$ и найти её КНФ $$\Psi = D_1 \land \dots \land D_m.$$
    \item Найти резолютивный вывод значения 0 из множества $S = \{D_1, \dots, D_m\}$.
    \item Если такой вывод существует, то выполняется $\Phi_1, \dots, \Phi_n \vDash \Phi$.
\end{enumerate}