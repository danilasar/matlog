\section{Логическая равносильность}
\subsection{Интерпретация формул алгебры предикатов}
\dftion \textbf{Область интерпретации} --- непустое множество $M$, которое является областью возможных значений всех предметных переменных.
\dftion \textbf{Оценка} предметных переменных --- $\alpha: X \to M$, где $X$ --- множество всех предметных переменных, $M$ --- область интерпретации.
\dftion \textbf{Выполнимость} формулы $\Phi$ в интерпретации $M$ при оценке $\alpha$ обозначается $M \vDash_\alpha \Phi$ и означает, что формула $\Phi$ истинна в интерпретации $M$ при оценке $\alpha$.
\subsection{Тавтологии алгебры предикатов}
\dftion  $\Phi$ является:
\begin{itemize}
    \item \textit{общезначимой} (тождественно истинной, общезначимой), если $M \vDash_\alpha \Phi$ в любых интерпретациях $M$ при любых оценках $\alpha$. обозначается как $\vDash \Phi$;
    \item \textit{выполнимой}, если $M \vDash_\alpha \Phi$ в некоторой $M$ при некоторой $\alpha$
    \item \textit{опровержимой}, если в некоторой $M$ при некоторой $\alpha$ неверно, что $M \vDash_\alpha \Phi$
    \item \textit{тождественно ложной}, если в любой $M$ для любой $\alpha$ неверно, что $M \vDash_\alpha \Phi$
\end{itemize}

Любая тавтология алгебры высказываний ялвяется тавтологией алгебры предикатов.

\textbf{Лемма 1}. Если $\Phi(X_1, \dots, X_n)$ --- тавтология алгебры высказываний, то для любых формул алгебры предикатов $\Phi_1, \dots, \Phi_n$ формула $\Phi(\Phi_1, \dots, \Phi_n)$ является тавтологией алгебры предикатов.

\textbf{Лемма 2}. Для любых формул $\Phi, \Psi$ следующие формулы являются тавтологиями:
\begin{enumerate}
    \item $\lnot(\forall x)\Phi \eq (\exists x)(\lnot \Phi)$ \\
    $\lnot(\exists x)\Phi \eq (\forall x)(\lnot x)$

    \item $(\forall x)(\forall y) \eq (\forall y)(\forall x)$ \\
    $(\exists x)(\forall y) \then (\forall y)(\exists x)$

    \item $(\forall x)(\Phi \land \Psi) \eq (\forall x)\Phi \land (\forall x)\Psi$

    $(\exists x)(\Phi \lor \Psi) \eq (\exists x)\Phi \lor (\exists x)\Psi$ \\

    \item Если $\pi \in \{\land, \lor\}$, а $x$ в $\Psi$ не входит свободно: \\
    $(\forall x)(\Phi \pi \Psi) \eq (\forall x)\Phi \pi \Psi$ \\
    $(\exists x)(\Phi \pi \Psi) \eq (\exists x)\Phi \pi \Psi$
\end{enumerate}
\subsection{Логическая равносильность формул алгебры предикатов}
\dftion Формулы алгебры предикатов $\Phi, \Psi$ называются логически равносильными ($\Phi = \Psi$, $\Phi \eq \Psi$) тогда и только тогда, когда $\vDash \Phi \eq \Psi$.

\textbf{Теорема 1 (взаимосвязь между кванторами)}.
$$(\forall x)(\forall y)\Phi = (\forall y)(\forall x)$$
$$(\exists x)(\exists y)\Phi = (\exists y)(\exists x)$$
Если $x, y$ не входят в $\Phi$ свободно, то $(\forall x)(\exists y)\Phi = (\exists y)(\forall x)\Phi$

\textbf{Теорема 2}. Имя предметной переменной при кванторе можно поменять при условии, что переменной с новым именем не содержится в формуле $\Phi$.

\textbf{Теорема 3 (законы де Моргана для кванторов)}.
$$\lnot (\forall x) \Phi = (\exists x)\lnot \Phi$$
$$\lnot (\exists x) \Phi = (\forall x)\lnot \Phi$$

\textbf{Теорема 4 (взаимосвязь кванторов с конхюнкцией и дизъюнкцией)}.
$$(\forall x)(\Phi \land \Psi) \eq (\forall x)\Phi \land (\forall x)\Psi$$
$$(\exists x)(\Phi \lor \Psi) \eq (\exists x)\Phi \lor (\exists x)\Psi$$

\textbf{Теорема 5 (взаимосвязь кванторов с импликацией)}. Если в $\Phi$ предметная переменная $x$ не входит свободно, то
$$(\forall x)(\Phi \then \Psi) = \Phi \then (\forall x)\Psi$$
$$(\exists x)(\Phi \then \Psi) = \Phi \then (\exists x)\Psi$$
Если в $\Psi$ предметная переменная $x$ не входит свободно, то
$$(\forall x)(\Phi \then \Psi) = (\exists x)\Phi \then \Psi$$
$$(\exists x)(\Phi \then \Psi) = (\forall x)\Phi \then \Psi$$


\subsection{Свойства логических операций над предикатами}
\textit{Почему-то материал отсутствует в конспектах\dots}
\subsection{Логическое следование формул алгебры предикатов}
\dftion Формула $\Phi$ алгебры предикатов называется \textbf{логическим следствием} формулы $\Psi$, если $\vDash \Phi \then \Psi$, т. е. в любой интерпретации $M$ формула истинна при любой оценке переменных $\alpha$, при которой истинна $\Psi$.

\dftion Формула $\Phi$ называется \textbf{логичесим следствием} множеста $\Gamma$, если в любой интерпретации $M$ формула $\Phi$ истинна при любой переменных $\alpha$, если истинны все формулы из $\Gamma$: $\Gamma \vDash \Phi$.

\underline{Лемма (критерии логического следования)}. Условие $\xses[\Phi] \vDash$ равносильно каждому из следующих условий:
\begin{enumerate}
    \item $\Phi_1 \land \dots \land \Phi_n \vDash \Phi$
    \item $\vDash \Phi_1 \land \dots \land \Phi_n \then \Phi$
    \item $\Phi_1 \land \dots \land \Phi_n \land \lnot \Phi \vDash$
\end{enumerate}