\section{Предварительная нормальная форма (ПНФ) формул алгебры предикатов}
\dftion \textbf{Предварённая (пренексная) нормальная форма} (ПНФ) --- формула вида $(K_1 x_1)\dots(K_n x_n)\Psi$, где $\xses[K]$ --- кванторы, $\Psi$ --- бескванторная формула в КНФ. Последовательность кванторов называется \textit{кванторной приставкой}, а $\Psi$ --- \textit{конъюнктивным ядром} формулы $\Phi$.

\underline{Теорема 1}. Любая формула перечисления предикатов $\Phi$ логически равносильна формуле $\Phi'$ в ПНФ, причём $\Phi'$ называется \textbf{пренексной нормальной формулой} формулы $\Phi$.