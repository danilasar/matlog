\section{Скулемовская стандартная формула (ССФ) формул алгебры предикатов}
\dftion \textbf{Скулемовская стандартная форма} (ССФ) --- частный случай ПНФ, при котором в кванторной приставке содержатся только кванторы $\forall$. Приведение к ССФ осуществляется согласн стандартному алгоритму:
\begin{enumerate}
    \item Если левее $(\exists x)$ не стоит кванторов общности, $x$ в конъюнктивном ядре заменяется новым предметным символом $c$, а квантор убирается
    \item Если же там стоят кванторы $(\forall x_{s_1})\dots(\forall x_{s_m})$, все вхождения заменя.ься $x$ на новый $m$-арный функциональный символ $f(x_{s_1},\dots,x_{s_m})$ и убирается квантор $(\exists x)$
\end{enumerate}