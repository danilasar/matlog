\section{Формулы и истинностные значения формул}
\dftion \textbf{Пропозициональные формулы} --- это формулы, с помощью котороых описываются свойства алгебры высказываний $\mathscr{P}$. \textbf{Пропозициональные связки} --- это символы логических операций в них, а \textbf{пропозициональные переменные} --- это символы $X, Y, Z, \dots$, которые используются для обозначения высказываний.

\dftion Формулы алгебры логики образуют множество $\mathfrak I$ и индуктивно определяются по правилам:
\begin{enumerate}
    \item Каждая пропозициональная переменная является формулой
    \item Если $\Phi, \Psi$ --- формулы, то формулами являются и выражения операций над ними.
\end{enumerate}

\dftion Приоритет логических операций по убыванию: отрицание, конъюнкция, дизъюнкция, а также следование и равносильность, имеющие равный приоритет.

\textbf{Лемма об основных равенствах формул.} \label{q2:lemma}
2 безымянных, следование, равносильность, сочетательный/перестановочный закон, де Морган, поглощение и двойное отрицание (всего 9):
\begin{enumerate}
    \item $A \lor (B \lor C) = (A \lor B) \lor C$ \Big($A \land (B \land C) = (A \land B) \land C$\Big) --- сочетательный закон;
    \item $A \lor B = B \lor A$ \Big($A \land B = B \land A$\Big) --- перестановочный закон;
    \item $A \lor (B \land C) = (A \lor B) \land (A \lor C)$ \Big($A \land (B \lor C) = (A \land B) \lor (A \land C)$\Big);
    \item $X \land X = X$ \Big($X \lor X = X$\Big);
    \item $\lnot(A \land B) = \lnot A \lor \lnot B$ \Big($\lnot(A \lor B) = \lnot A \land \lnot B$\Big) --- закон де Моргана;
    \item $X \lor (X \land Y) = X$ \Big($X \land (X \lor Y) = X$\Big) --- закон поглощения;
    \item $\lnot \lnot X = X$ --- закон двойного отрицания;
    \item $X \then Y = \lnot X \lor Y = \lnot (X \land \lnot Y)$
    \item $X \eq Y = (X \land Y) \lor (\lnot X \land \lnot Y) = (X \lor \lnot Y) \land (\lnot X \lor Y)$
\end{enumerate}

\dftion \textbf{Истинностное значение} $\lambda(A)$ ставит в соответствие высказыванию 1, если оно истинно, или 0, если оно ложно.

\dftion Если в формулу $\Phi$ входят переменные $X_1, \dots, X_n$, то записывают $\Phi = \Phi(X_1, \dots, X_n)$. Формула определяет функцию $n$ переменных $F$, которая каждому упорядоченному набору $(\lambda(X_1), \dots, \lambda(X_n))$ из $n$ элементов множества $\{0, 1\}$ ставит в соответствие $\lambda(\Phi(X_1, \dots, X_n))$ этого же множества.
