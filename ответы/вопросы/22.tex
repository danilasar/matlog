\section{Аксиомы и правила вывода исчисления предикатов. Тождественная истинность выводимых формул}
Множество $Ax(\text{ИП})$ содержит в себе пять аксиом:

\begin{enumerate}
    \item $(A_1)$: $\Phi \then (\Psi \then \Phi)$
    \item $(A_2)$: $(\Phi \then (\Psi \then \Xi)) \then ((\Phi \then \Psi) \then (\Psi \then \Xi))$
    \item $(A_3)$: $(\lnot \Phi \then \lnot \Psi) \then ((\lnot \Phi \then \Psi) \then \Phi)$
    \item $(A_4)$: $(\forall x)\Phi(x) \then \Phi(y)$ для формул $\Phi(x)$, в которые $x$ не входит связно
    \item $(A_5)$: $(\forall x)(\Phi \then \Psi(x)) \then (\Phi \then (\forall x)\Psi(x))$ для формул $\Phi$, в которые $x$ не входит свободно
\end{enumerate}

Исчисление предикатов имеет два правила вывода --- \textbf{правило заключения} (modus ponens) и \textbf{правило обобщения}:

\begin{figure}[H]
    \centering
    \begin{tabular*}{0.5\textwidth}{@{\extracolsep{\fill}}ccc@{}}
        $MP: \begin{matrix}
            \Phi \then \Psi, \Phi \\
            \Psi
        \end{matrix}$ &
        и &
        $Gen: \begin{matrix}
            \Phi \\
            (\forall x)\Phi
        \end{matrix}$.
    \end{tabular*}
\end{figure}