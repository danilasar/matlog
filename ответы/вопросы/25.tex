\section{Разрешимые и полуразрешимые языки. Теорема Поста}

\underline{Определение 1}. Язык $L$ называется \textit{разрешимым} (или \textit{рекурсивным}), если существует такая машина Тьюринга $T$, что для любого слова $w \in W$ выполняются условия:
\begin{itemize}
    \item если $w \in L$, то при входе $w$ машина $T$ попадает в заключительное состояние, останавливается и выдаёт значение $T(w) = 1$
    \item если $w \not \in L$, то при входе $w$ машина $T$ попадает в заключительное состояние, останавливаетс и выдаёт значение $T(w) = 0$
\end{itemize}

Такие машины соответствуют понятию <<алгоритма>> и применяются при решении \textit{распознавательных задач} типа <<да/нет>>.

Множество всех разрешимых языков будем обозначать $R$ (от Recursive).

\textbf{Свойства}: дополнения, конечные пересечения и конечные объединения разрешимыъх языков являются разрешимыми языками.

\underline{Определение 2}. Язык $L$ называется \textit{полуразрешимым} или \textit{перечислимым}, если существует такая машина Тьюринга, что
\begin{equation}
    L = L(T) = \{w \in \Sigma* : T(w) = 1\},
\end{equation}
то есть при выходе $w \in L$ машина $T$ попадает в заключительное состояние, останавливается и выдаёт значение $T(w) = 1$, а при выходе $w \not \in L$\dots\dots\dots

\underline{Лемма}. Существуют неразрешимые языки, поскольку алгоритмов счётное число, а языков несчётное.

Аналогично можно доказать, что существуют языки, не являющиеся полуразрешимыми.

\underline{Основная теорема}. Существуют полуразрешимые неразрешимые языки, т. е. полуразрешимые языки, которые не могут быть разрешимы никакм алгоритмом, т. е. выполняется свойство $R \not \subset RE$.

\textbf{Теорема Поста} утверждает, что не существует общего алгоритма, который мог бы решить любую возможную задачу в области математики. То есть, не существует универсального алгоритма, способного определить принадлежность любой строки к произвольному языку.