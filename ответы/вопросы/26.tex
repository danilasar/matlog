\section{Машины Тьюринга и вычислимые ими функции}
\dftion \textbf{Машина Тьюринга} $T$ представляет собой алгебраическую систему $T = (\Sigma, Q, \delta, q_S, q_F)$, работающую в дискретные моменты времени $t=0,1,2\dots$ и состоящую из следующих частей:

\begin{itemize}
    \item Конечное множество $\Sigma=\{0,1\dots\}$ называется \textit{внешним алфавитом}
    \item Конечное множество $Q = \{q_S, q_F, \dots\}$ называется \textit{внутренним алфавитом}, элементы $Q$ называются \textit{состояниями машины}
    \item Отображение $\delta: Q \times \Sigma \to Q \times \Sigma \times \{R, L, S\}$, которое определяет список команд $T(q, a) = qa \to q'a'X$ --- символическое оюозначение образов $\delta(q, a) = (q', a', X)$ отображения $\delta$ для $q \in Q \backslash \{q_F\}, a \in \Sigma$ и $X \in \{R, L, S\}$, множество всех команд $\Pi = \{T(q, a): q \in Q \backslash \{q_F\} \land a \in Sigma\}$ называется \textit{программой машины}
    \item Состояние $q_S$ называется \textit{начальным} и означает начало работы машины
    \item Состояние $q_F$ называется \textit{заключительным} и означает завершение работы машины
\end{itemize}

Помимо детерменированной машины Тьюринга $T=(\Sigma, Q, \delta, q_S, q_F)$ с одной программой $\delta$ в теории алгоритмов рассматриваются \textit{недетерменированные машины Тьюринга} $T=(\Sigma, Q, \delta_1, \delta_2, q_S, q_F)$ с двумя программами $\delta_1, \delta_2$, которая на каждом шаге случайным образом выбирает одну из этих двух программ и по ней выполняет измеение своей конфигурации.

\dftion \textbf{Вычислимая функция} (или алгоритмически вычислимая функция) — это функция, для которой существует алгоритм, который может её вычислить. Машина Тьюринга используется для формализации понятия алгоритма, и любая вычислимая функция может быть вычислена с помощью машины Тьюринга.