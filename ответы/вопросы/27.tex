\section{Распознаваемость языков машинами Тьюринга}
\underline{Определение 1}. Язык $L$ называется \textit{разрешимым} (или \textit{рекурсивным}), если существует такая машина Тьюринга $T$, что для любого слова $w \in W$ выполняются условия:
\begin{itemize}
    \item если $w \in L$, то при входе $w$ машина $T$ попадает в заключительное состояние, останавливается и выдаёт значение $T(w) = 1$
    \item если $w \not \in L$, то при входе $w$ машина $T$ попадает в заключительное состояние, останавливается и выдаёт значение $T(w) = 0$
\end{itemize}

Такие машины соответствуют понятию <<алгоритма>> и применяются при решении \textit{распознавательных задач} типа <<да/нет>>.

Множество всех задач будем обозначать $R$ (от Recursive).

\textbf{Свойства}: дополнения, конечные пересечения и конечные объединения разрешимых языков называются рарешимыми языками.

\underline{Определение 2}. Язык $L$ называется \textit{полуразрешимым} или \textit{перечислимым}, если существует такая машина Тьюринга, что

\begin{enumerate}
    \item При входе $w \in L$ машина $T$ попадает в заключительное состояние, останавливается и выдаёт значение $T(w) = 1$
    \item При входе $w \not\in L$ машина Тьюринга $T$ не даёт никакого результата
\end{enumerate}
