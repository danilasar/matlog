\section{Тавтологии. Методы доказательства тавтологий}
\dftion Формула $\Phi$ называется
\begin{itemize}
    \item \textbf{тавтологией} (тождественно истинной), если она истинна при любых значениях переменных, и обозначается $\vDash \Phi$;
    \item \textit{выполнимой}, если найдутся такие значения переменных, при которых $\Phi$ является истинной;
    \item \textit{опровержимой}, если найдутся такие значения переменных, при которых $\Phi$ является ложной;
    \item \textbf{противоречием} (тождественно ложной), если она ложна (или противоречива) при любых значениях переменных.
\end{itemize}
\dftion Тавтологии являтся общими схемами построения логических высказываний и выражают \textbf{логические законы}.
\begin{enumerate}
    \item Прямой метод подразумевает перебор всех возможных значений переменных и построение таблицы истинности.
    \item Алгебраический метод предполагает приведение формулы $\Phi$ к тождественно истинной формуле $1$ путём равносильных преобразований.
    \item Метод Квайна предполагает построение дерева решений с подстановкой возможных значений очередной переменной на каждой очередной высоте дерева.
    \item Метод редукции основывается на движении от обратного: истинностное значение формулы $\Phi$ принимается за 0 и, на основании этого, восстанавливаются значения исходных переменных. Если возникает противоречие, формула является тождественно истинной. Особенно удобен в формулах с большим количеством импликаций благодаря её свойству: она ложна тогда и только тогда, когда посылка истинна, а заключение ложно.
    \item Метод семантических таблиц
    \item Метод резолюций сводит доказательство тавтологии к доказательству противоречивости обратного путём резолютивного вывода нуля
\end{enumerate}