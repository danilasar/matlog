\section{Классы языков $\mathscr{P}$ и $\mathscr{NP}$}

Говорят, что машина Тьюринга $T$ имеет \textbf{полиномиальную временную сложность} $P(n)=n^k$ (<<время работы ограничено полиномом $P(N)$>>), если, обрабатывая вход $w$ длины $n$, $T$ делает не более $P(n)$ переходов и останавливается независимо от того, допущен вход или нет.

\dftion Говорят, что язык $L$ принадлежит классу $\mathscr{P}$, если он разрешим некоторой детерминированной машины Тьюринга $T$ с полиномиальной временной сложностью.

\dftion Распознавательная задача принадлежит классу $P$, если её язык принадлежит классу $\mathscr{P}$, то есть эта задача решается с помощью полиномиального алгоритма --- некоторой детерменированной машины Тьюринга $T$ с полиномиальной временной сложностью.

Помимо детерменированной машины Тьюринга $T=(\Sigma, Q, \delta, q_S, q_F)$ с одной программой $\delta$ в теории алгоритмов рассматриваются \textit{недетерменированные машины Тьюринга} $T=(\Sigma, Q, \delta_1, \delta_2, q_S, q_F)$ с двумя программами $\delta_1, \delta_2$, которая на каждом шаге случайным образом выбирает одну из этих двух программ и по ней выполняет измеение своей конфигурации.

\dftion Язык $L$ принадлежит классу $\mathscr{NP}$, если он разрешим некоторой недетерменированной машиной Тьюринга $T$ с полиномиальной временной сложностью.