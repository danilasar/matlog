\section{Понятие предиката и его множества истинности. Перенесение на предикаты логических операций}
\dftion \textbf{Предикат} --- утверждение, содержащее \textit{предикативные} переменные $x_1,\dots,x_n$, которое превращается в высказывание при замене переменных конкретными объектами из области возможных значений $M$. Предикат с $n$ переменными называется $n$-арным или $n$-местными и обозначается $P(x_1,\dots,x_n)$.

Истинностная функция предиката $F_P : M^n \to \{0,1\}$ определяется множеством истинности $P^+ = \{(a_1, \dots, a_n) \in M^n : \lambda(P(a_1,\dots,a_n)) = 1\}$.

Перенесение на предикаты логических операций:
\dftion \textbf{Отрицание} $n$-местного предиката --- $n$-местный предикат, который при подстановке значений превращается в высказывание, являющееся отрицанием высказывания исходного предиката.

\dftion \textbf{Конъюнкция} (\textbf{дизъюнкция}) $n$-местных предикатов $P(x_1,\dots,$ $x_n)$, $Q(x_1,\dots,x_n)$ --- $n$-местный предикат, который при подстановке значений превращается в высказывание $P\land Q(x_1,\dots,x_n)$ \Big($P\lor Q(x_1,\dots,x_n)$\Big).

Множество истинности предикатов, полученных при помощи логических операций:
\begin{itemize}
    \item $(\lnot P)^+ = M \backslash P^+$
    \item $(P \land Q)^+ = P^+ \cap Q^+$
    \item $(P \lor Q)^+ = P^+ \cup Q^+$
    \item $(P \then Q)^+ = (\lnot P)^+ \cup Q^+$
    \item $(P \eq Q)^+ = (P \then Q)^+ \cap (Q \then P)^+$
\end{itemize}
